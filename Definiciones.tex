\section{Definiciones básicas}

\subsection{Población, muestra y unidad elemental}

La población se define como el conjunto total de elementos o individuos que poseen las características de interés para el estudio. En este contexto, la población está compuesta por [describir la población].

La muestra es un subconjunto representativo de la población, seleccionado de manera que permita inferir propiedades sobre la población total. La muestra utilizada en este estudio consta de [número] elementos.

La unidad elemental es el individuo o elemento básico sobre el cual se recopilan los datos. En este caso, la unidad elemental es [describir, e.g., cada estudiante encuestado].

\subsection{Definición y clasificación de variables utilizadas}

Las variables utilizadas en el estudio se clasifican de la siguiente manera:

\subsubsection{Variables cualitativas}
- Variable 1: Descripción.
- Variable 2: Descripción.

\subsubsection{Variables cuantitativas}
- Variable 3: Descripción (tipo: discreta/continua).
- Variable 4: Descripción (tipo: discreta/continua).
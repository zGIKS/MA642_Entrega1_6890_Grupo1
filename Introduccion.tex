\section{Introducción}

El comercio internacional ha experimentado transformaciones significativas en los últimos años, impulsadas por el aumento del proteccionismo y la volatilidad económica global. Según Aráoz (2025), en el contexto de la reactivación de medidas arancelarias por parte de Estados Unidos y el fortalecimiento de barreras comerciales frente a economías estratégicas como China, ``los tratados de libre comercio, incluyendo los de América Latina, enfrentan pruebas críticas ante un entorno de incertidumbre global, tensiones geopolíticas y cambios en los flujos comerciales tradicionales''. Estas dinámicas han generado nuevos retos para las economías emergentes, entre ellas el Perú, que debe identificar oportunidades y estrategias para sostener su competitividad en el mercado internacional.

En este escenario, la palta Hass peruana se ha consolidado como uno de los principales productos agroexportadores del país. Obando Cazorla (2025) destaca que la palta Hass constituye un motor clave del crecimiento exportador nacional, demostrando un incremento sostenido en volumen, valor y posicionamiento en mercados estratégicos durante las últimas dos décadas. América del Norte ---compuesta por Estados Unidos, Canadá y México--- representa uno de los destinos más relevantes para esta fruta, impulsado por una demanda creciente, acuerdos comerciales favorables y un mercado orientado a productos saludables.

Por lo tanto, el presente estudio tiene como objetivo analizar las exportaciones de palta desde Perú hacia América del Norte durante el año 2025, utilizando información obtenida del portal ADEX Data Trade. Para ello, se aplican técnicas de estadística descriptiva mediante el uso de Excel y R con el fin de organizar, procesar y representar los datos a través de tablas y gráficos, facilitando la identificación de patrones comerciales y tendencias exportadoras.

El análisis se enfocará en variables relevantes para comprender el comportamiento exportador, tales como país de destino, vía de transporte, registros de exportación y valor FOB. A partir de ello, se busca proporcionar una visión detallada de la dinámica del comercio exterior de palta peruana en el contexto actual, contribuyendo al entendimiento de los desafíos y oportunidades para el sector agroexportador nacional.

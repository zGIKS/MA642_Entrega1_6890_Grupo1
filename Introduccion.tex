\section{Introducción}

El comercio internacional enfrenta un escenario marcado por tensiones geopolíticas, reacomodos estratégicos y un incremento del proteccionismo económico. Según Aráoz (2024), ``El año 2025 se presenta como un año de incertidumbre económica global'', en el que ``el reacomodo geopolítico y geoeconómico será el elemento que marcará las relaciones económicas internacionales''. Además, la autora advierte que las nuevas políticas comerciales y medidas arancelarias en Estados Unidos frente a economías como China, Canadá y México conducirán a ``disrupciones en las cadenas de valor globales'', acompañadas de un incremento del ``nacionalismo (proteccionismo y política industrial) en los países desarrollados''.

En este contexto, el Perú se posiciona como un actor relevante en el comercio agrícola internacional, destacando la exportación de palta Hass como uno de los principales motores del sector agroexportador. Obando Cazorla (2025) afirma que ``la palta Hass constituye uno de los principales productos agrícolas de exportación del Perú, destacando por su dinamismo y sostenido crecimiento en los mercados internacionales''. La creciente demanda en mercados como Estados Unidos, Canadá y México, junto con acuerdos comerciales favorables, ha impulsado su expansión y competitividad global.

Ante este panorama internacional, se vuelve fundamental analizar el comportamiento de las exportaciones peruanas de palta hacia América del Norte en el año 2025. El presente estudio emplea herramientas de estadística descriptiva mediante el uso de Excel y R, utilizando información oficial del portal ADEX Data Trade. El análisis permitirá organizar, procesar y representar los datos en tablas y gráficos, con el objetivo de identificar patrones comerciales, modalidades de transporte y tendencias relevantes que contribuyan a comprender la dinámica exportadora peruana en un entorno global desafiante.
